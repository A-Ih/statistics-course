\documentclass[12pt,a4paper]{article}
\usepackage[a4paper, margin={0.5in}]{geometry}
\usepackage[utf8]{inputenc}
\usepackage[russian]{babel}
\usepackage{amsmath}
\usepackage{amsfonts}
\usepackage{amssymb}
% \usepackage{makeidx}
% \usepackage{cancel}
% \usepackage{hyperref}
% \usepackage{graphicx}
% \usepackage{xcolor}

\begin{document}
\begin{center}
\large Лабораторная 1 \\
\normalsize Вариант 1 (Задание 1)\\
\normalsize Вариант 3 (Задание 2)\\
\normalsize Ибрахим Ахмад M33391 \\
\end{center}
%\maketitle
%\pagebreak
% \center
\section*{Задание 1}
\par $X_{1} \ldots X_{n} \sim Bern(p)$, нужно найти минимальное $n$, при котором
$P(|EX_{1}-\overline{X}| \geq \epsilon) \leq \delta$, где $\epsilon=0.1$ и
$\delta=0.05$.

\par Вспомним условия ЦПТ:
\par $\{Y_{n}\}$~--- i.i.d., $EY_{i}=\mu<\infty$, $DY_{i}=\sigma^2<\infty$;
    $Z_{n}=\frac{S_{n}-n\mu}{\sigma \sqrt{n}}$, $Z \sim N(0,1)$.
    Тогда $Z_{n} \xrightarrow{d} Z$

\par Попытаемся преобразовать $P(|EX_{1}-\overline{X}| \leq x)$ так, чтобы можно
было выразить это в терминах ЦПТ. Положим $Y_{i}=X_{i}$ в условии теоремы.
$$P(|\underbrace{EX_{1}}_{\mu}-\underbrace{\overline{X}}_{S_{n}/n}| \leq x)
  =P(|n\mu-S_{n}| \leq nx)
  =P\left(\left|\frac{n\mu-S_{n}}{\sigma \sqrt{n}}\right| \leq \frac{x
      \sqrt{n}}{\sigma}\right)$$
$$=F_{Z_n}\left(\frac{x \sqrt{n}}{\sigma}\right)-F_{Z_n}\left(-\frac{x \sqrt{n}}{\sigma}\right)$$
$$\approx \Phi\left(\frac{x \sqrt{n}}{\sigma}\right) - \Phi\left(-\frac{x \sqrt{n}}{\sigma}\right)$$
$$=2\Phi\left(\frac{x \sqrt{n}}{\sigma}\right) - 1$$

Отлично, теперь выразим через это приближение исходное условие:
$$P(|EX_{1}-\overline{X}| \geq \epsilon) \leq \delta$$
$$1 - P(|EX_{1}-\overline{X}| \leq \epsilon) \leq \delta$$
$$P(|EX_{1}-\overline{X}| \leq \epsilon) \geq 1 - \delta$$
$$2\Phi\left(\frac{\epsilon \sqrt{n}}{\sigma}\right) - 1 \geq 1 - \delta$$
$$\Phi\left(\frac{\epsilon \sqrt{n}}{\sigma}\right) \geq 1 - \delta/2$$
По строгому возрастанию  $\Phi$, можем применить $\Phi^{-1}$ к неравенству:
$$\frac{\epsilon \sqrt{n}}{\sigma} \geq \Phi^{-1}\left(1 - \delta/2\right)$$
$$\sqrt{n} \geq \frac{\sigma}{\epsilon} \Phi^{-1}\left(1 - \delta/2\right)$$
$$n \geq \left(\frac{\sigma}{\epsilon} \Phi^{-1}\left(1 - \delta/2\right)\right)^2$$
Т.к. мы пытаемся найти наименьшее натуральное $n$, то можно явно выразить его из
неравенства выше:
$$n = \left\lceil\left(\frac{\sigma}{\epsilon} \Phi^{-1}\left(1 -
    \delta/2\right)\right)^2\right\rceil$$
Вспомним, что $\mu=p$, $\sigma^2=p-p^2$ и подставим в формулу $p=0.4$,
$\epsilon=0.1$ и $\delta=0.05$:
\begin{verbatim}
>>> from scipy.stats import norm
>>> p = 0.4
>>> eps = 0.1
>>> delta = 0.05
>>> mu = p
>>> sigma = math.sqrt(p - p*p)
>>> n = ((sigma/eps)*norm.ppf(1 - delta/2))**2
>>> print(n)
92.195011696659
\end{verbatim}

$$n=\lceil 92.195011696659 \rceil=93$$
Ответ: 93. \\
\par Анализ выборок приведён в \texttt{task-1.ipynb}
\section*{Задание 2}
Решение описано в \texttt{task-2.ipynb}
\end{document}

